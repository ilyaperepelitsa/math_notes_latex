\documentclass{article}  % or another class like amsart, report, book, etc.

% Preamble: load necessary packages
\usepackage{amsmath, amsthm, amssymb}  % Essential for math environments and symbols
\usepackage{booktabs} % Provides nicer table rules
\usepackage{tikz}
\usepackage{amssymb}
% Optional: Define theorem environments
\newtheorem{theorem}{Theorem}[section]
\newtheorem{lemma}[theorem]{Lemma}
\theoremstyle{definition}
\newtheorem{definition}[theorem]{Definition}

% Begin document
\begin{document}
\section{Definitions}
Union and intersection of A and B:
\[
\begin{aligned}
    A \cup B &= \{x \mid x \in A \lor x \in B\} \\
    A \cap B &= \{x \mid x \in A \land x \in B\} \\
\end{aligned}
\]
Union and intersection for three or more sets:
\[
\begin{aligned}
    A \cup B \cup C &= \{x \mid x \in A \lor x \in B \lor x \in C\} \\
    A \cap B \cap C&= \{x \mid x \in A \land x \in B \land x \in C\} \\
\end{aligned}
\]
Disjoint set:
\[
\begin{aligned}
    A \cap B &= \varnothing
\end{aligned}
\]
Union from 1 to n of A sub k
\[
\begin{aligned}
    \bigcup_{i=1}^{n} A_i &= A_{1} \cup A_{2} \cup \cdots \cup A_{n}
\end{aligned}
\]
Complement of B with respect to A:
\[
\begin{aligned}
    A - B &= \{x \mid x \in A \land x \notin B\} \\
\end{aligned}
\]
Complement of A:
\[
\begin{aligned}
   \overline{A} &= \{x \mid x\notin A\}
\end{aligned}
\]
Symmetric difference:
\[
\begin{aligned}
    A \oplus B &= \{x \mid (x \in A \land x \notin B) \lor (x \in B \land x \notin A)\}
\end{aligned}
\]


\section{Theorem: Algebraic properties of Set Operations}
\subsection{Commutative Properties}
\begin{theorem}
\[
    A \cup B = B \cup A
\]
\end{theorem}
\begin{proof}

    By definition
    \[
        A\cup B = \{x \mid x \in A \:\lor\: x \in B\}
    \]

    consists of all elements that belong to A or B\\
    if x is a member in either A or B, with commutative property the only thing
    that changes is order of checking whether $x \in A \lor \: x \in B$
\end{proof}
\begin{proof}
    Bad way to prove by bruteforcing
    \begin{table}[ht]
    \centering
    \begin{tabular}{cccc}
        \toprule
        \( x \in A \) & \( x \in B \) & \((x \in A) \lor (x \in B)\) & \((x \in B) \lor (x \in A)\) \\
        \midrule
        True  & True  & True  & True  \\
        True  & False & True  & True  \\
        False & True  & True  & True  \\
        False & False & False & False \\
        \bottomrule
    \end{tabular}
    \caption{Truth Table Demonstrating the Commutativity of \(\lor\) in Set Union}
    \end{table}

\end{proof}

\begin{theorem}
    \[
        A \cap B = B \cap A
    \]
\end{theorem}

\begin{proof}
        By definition
    \[
        A\cap B = \{x \mid x \in A \:\land\: x \in B\}
    \]

    Both conditions must be satisfied.
    If $x \in B$ and $x \in A$ implies that both conditions
    have to be satisfied.

\end{proof}

\subsection{Associative Properties}
\begin{theorem}
\[
    A\cup(B\cup C) = (A\cup B) \cup C
\]
\end{theorem}
\begin{proof}

\[
\begin{aligned}
    B \cup C &= \{x \mid x \in B \lor x \in C \} \\
    A \cup (B \cup C) &= \{x \mid x \in A \lor (x \in B \lor x \in C)\} \\
    A \cup (B \cup C) &= \{x \mid (x \in A \lor x \in B) \lor x \in C\} \\
    A \cup B \cup C &= \{x \mid x \in A \lor x \in B \lor x \in C\}
\end{aligned}
\]
Hence the result follows.
\end{proof}


\begin{theorem}
\[
    A\cap(B\cap C) = (A\cap B) \cap C
\]
\end{theorem}
\begin{proof}

\[
\begin{aligned}
    B \cap C &= \{x \mid x \in B \land x \in C \} \\
    A \cap (B \cap C) &= \{x \mid x \in A \land (x \in B \land x \in C)\} \\
    A \cap (B \cap C) &= \{x \mid (x \in A \land x \in B) \land x \in C\} \\
    A \cap B \cap C &= \{x \mid x \in A \land x \in B \land x \in C\}
\end{aligned}
\]
Hence the result follows.
\end{proof}

\subsection{Distributive Properties}
\begin{theorem}
    \[
        A \cap (B \cup C) = (A \cap B) \cup (A \cap C)
    \]
\end{theorem}

\begin{proof}
\[
\begin{aligned}
    B\cup C &= \{x \mid x \in B \lor x \in C \} \\
    A\cap (B\cup C) &= \{x| (x \in B \lor x \in C ) \land x \in A\} = (A \cap B) \cup (A \cap C)\\\\
    (A \cap B) \cup (A \cap C)&=\{x \mid x \in (A \cap B) \lor x \in (A \cap C)\}\\
    x \in A \cap B &= x \in A \cap x \in B\\
    x \in A \cap C &= x \in A \cap x \in C
\end{aligned}
\]
Hence the result follows.
\end{proof}
\begin{theorem}
    \[
        A \cup (B \cap C) = (A \cup B) \cap (A \cup C)
    \]
\end{theorem}

\subsection{Idempotent Properties}
\begin{theorem}
    \[
        A \cup A = A
    \]
\end{theorem}
\begin{theorem}
    \[
        A \cap A = A
    \]
\end{theorem}

\subsection{Properties of the Complement}
\begin{theorem}
    \[
        (\overline{\overline{A}}) = A
    \]
\end{theorem}
\begin{theorem}
    \[
        A \cup {\overline{A}} = U
    \]
\end{theorem}
\begin{theorem}
    \[
        A \cap {\overline{A}} = \varnothing
    \]
\end{theorem}
\begin{theorem}
    \[
        \overline{\varnothing} = U
    \]
\end{theorem}

\begin{theorem}
    \[
        \overline{U} = \varnothing
    \]
\end{theorem}

\begin{theorem}
    \[
        \overline{A \cup B} = \overline{A} \cap \overline{B}
    \]
\end{theorem}

\begin{theorem}
    \[
        \overline{A \cap B} = \overline{A} \cup \overline{B}
    \]
\end{theorem}


\subsection{Properties of a Universal Set}

\begin{theorem}
    \[
        A \cup U = U
    \]
\end{theorem}


\begin{theorem}
    \[
        A \cap U = A
    \]
\end{theorem}

\subsection{Properties of the Empty Set}


\begin{theorem}
    \[
        A \cup \varnothing = A \lor A \cup \{\} = A
    \]
\end{theorem}

\begin{theorem}
    \[
        A \cap \varnothing = A \land A \cap \{\} = \{\}
    \]
\end{theorem}



\subsection{Example theorem solving}

\[
\begin{aligned}
    x &\in \overline{A \cup B}\\
    x &\notin A \cup B\\
    x &\notin A \land x \notin B \implies x \in \overline{A} \land x \in \overline{B} \\
    \overline{A \cup B} &\subseteq \overline{A} \cap \overline{B}\\
    \text{conversely } x &\in \overline{A} \cap \overline{B} \implies x \notin A \land x \notin B\\
    x &\notin A \cup B \implies x \in \overline{A \cup B}\\
    \{x \mid x \in \overline{A} \cap \overline{B}\} &= \{x \mid x \in \overline{A \cup B}\} \implies
        \overline{A \cup B} = \overline{A} \cap \overline{B}
\end{aligned}
\]

\section{Theorem:  Addition principle aka Inclusion-exclusion principle}

\begin{theorem}
If A and B are finite sets.
\[
    \begin{aligned}
        |A \cup B| = |A| + |B| - |A \cap B|\\
    \end{aligned}
\]
\end{theorem}

\begin{proof}
If \(A \cap B = \varnothing\), then each element of \(A \cup B\) is in either \(A\) or \(B\), but not both.
Hence \(\displaystyle |A \cup B| = |A| + |B|\).

On the other hand, if \(A \cap B \neq \varnothing\), then elements in \(A \cap B\) belong to both sets. Thus
\(\displaystyle |A| + |B|\) double-counts them, and we must subtract \(\displaystyle |A \cap B|\).

Hence the result follows.
\end{proof}


\begin{theorem}
If A , B and C are finite sets

\[
    \begin{aligned}
        |A \cup B \cup C| &= |A| + |B| + |C| - |A \cap B| - |B \cap C| - |A \cap C| + |A \cap B \cap C|
    \end{aligned}
\]
\end{theorem}

\end{document}
